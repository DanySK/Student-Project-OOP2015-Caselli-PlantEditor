\documentclass{article}
\usepackage[utf8]{inputenc}
\usepackage{hyperref}
\usepackage{listings}
\usepackage{enumitem}
\usepackage[italian]{babel}
\usepackage{graphicx}
\usepackage{framed}
\lstset{frame=single, breaklines=true}

\title{Corso di Laurea in Ingegneria e Scienze Informatiche\\\textbf{\newline\\Planted}\\{(OOP15 - Process Report)}}
\date{\today}
\author{Ashley Caselli (\href{mailto:ashley.caselli@studio.unibo.it}{ashley.caselli@studio.unibo.it})\\Matricola 635153}

\begin{document}
  \maketitle
  \newpage
  \tableofcontents
  \newpage
  %--------------------------------------------------------------------------------------------------------------------------------------
  \section{Introduzione}
  Questo documento descrive lo sviluppo software del progetto Planted, sviluppato per il corso di Programmazione ad Oggetti.
  %--------------------------------------------------------------------------------------------------------------------------------------
  \section{Requisiti}
  L’UML è un linguaggio di modellazione basato sul paradigma ad oggetti.\\Al momento esistono diversi strumenti software per la modellazione in UML.\\
  Il sistema software Planted sarà un semplice editor di testo che permetterà all'utente di scrivere codice conforme al linguaggio usato
  dalla libreria PlantUML (http://plantuml.com/), e mediante l'integrazione con essa, supporti la visualizzazione real time del modello che si sta creando.\\
  Inoltre il sistema software dovrà permettere l'esportazione del modello in un formato grafico (es. png).\\
  Nel caso in cui si stia scrivendo un modello UML dell classi, allora il sistema software dovrà permettere all'utente non solo di esportare il modello
  in formato grafico (infatti questa funzione deve essere possibile per ogni modello), ma anche di generare codice sorgente a partire dal modello delle classi
  che si sta definendo. L'utente quindi avrà la possibilità di scegliere il linguaggio nel quale vuole che il codice sorgente sia scritto.\\
  Il sistema software inoltre dovrà rendere possibile all'utente l'import di file al suo interno e la creazione del modello UML delle classi a partire
  da codice sorgente (es: scritto in Java).
  %--------------------------------------------------------------------------------------------------------------------------------------
  \section{Analisi}
  Il sistema software dovrà essere in grado di gestire progetti, al cui interno ci potranno essere più file di natura diversa (es: file che descrive il modello,
  file con codice sorgente, ecc...). Un \textbf{progetto (IProject)} è quindi un contenitore di file sorgenti che possono essere scritti in linguaggi diversi tra loro.
  Ogni progetto al suo interno potrà contenere dei file di natura diversa tra loro. All'interno dell'editor ci sarà la possibilità di creare più progetti,
  che a loro volta faranno parte dello stesso spazio di lavoro \textbf{(IWorkspace).}\\
  Un \textbf{file sorgente (ISourceFile)} presente all'interno di un progetto rappresenta perciò un file di testo con una certa estensione a seconda del linguaggio che contiene.\\
  Queste 3 entità appena descritte sono rappresentate nel modello UML sottostante:
  \begin{center}
    \includegraphics[scale=0.5]{img/IWorkspace}
    \includegraphics[scale=0.5]{img/IProject}
    \includegraphics[scale=0.5]{img/ISourceFile}
  \end{center}
  Ogni file sorgente sarà caratterizzato dal \textbf{tipo (FileType),} che rappresenterà la natura del suo contenuto e la sua estensione.
  \begin{center}
    \includegraphics[scale=0.5]{img/filetype}
  \end{center}
  Il sistema inoltre richiede la capacità di interpretare il contenuto testuale di un file per poter generare codice sorgente nel linguaggio scelto.
  Per soddisfare tale requisito sarà necessaria un'entità denominata \textbf{parser (IParser)}, che
  \begin{center}
    \includegraphics[scale=0.5]{img/IParser}
  \end{center}




  %--------------------------------------------------------------------------------------------------------------------------------------
  \section{Design}
  %--------------------------------------------------------------------------------------------------------------------------------------
  \subsection{Architettura}
  Per lo sviluppo del progetto si è scelto di utilizzare il pattern architetturale MVC, creando una suddivisione logica dei componenti come segue:
  \begin{itemize}
    \item \textbf{Model} si occupa dei dati del sistema, in particolare fornisce funzionalità per la persistenza dei dati all'interno del sistema e fornisce
    accesso ai dati al controller.
    \item \textbf{View} si occupa di fornire all'utente la possibilità di interagire con il sistema (a livello grafico), mostrando l'editor
    di testo, l'albero dei progetti all'interno del workspace e tutte le possibili azioni che l'utente può fare (comandi che saranno passati al controller)
    con il controller.
    \item \textbf{Controller} contiene la business logic del sistema. Si occupa dell'interazione tra model e view, in particolare riceve i comandi che l'utente da al sistema
    attraverso la view e li attua, aggiornando in seguito sia model che view.
  \end{itemize}
  \begin{framed}
    \centering
    \begin{center}
      \includegraphics[scale=0.5]{img/mvc}
    \end{center}
  \end{framed}
  %--------------------------------------------------------------------------------------------------------------------------------------
  \subsubsection{Model}
  Il livello di Model fornisce accesso ai dati del sistema alla business logic, che è rappresentata dal livello denominato Controller.\\
  I dati presenti all'interno del sistema sono definiti dal seguente modello dei dati:
  \begin{framed}
    \centering
    \textbf{Data Model}
    \begin{center}
      \includegraphics[scale=0.5]{img/datamodel}
    \end{center}
  \end{framed}
  Come si può notare dal modello dei dati viene utilizzato il \textbf{bridge pattern} per separare l'entità ISourceEntity dalla sua concreta
  implementazione (ISourceEntutyImpl). Questo permette di avere una maggiore libertà nel caso in cui si volessero aggiungere funzionalità
  al sistema in futuro.\\
  \\L'entità \textbf{IModel} rappresenta il punto di accesso al livello di Model per la business logic. Essa rappresenta quindi il vero e
  proprio livello di Model del sistema. Le funzionalità che essa fornisce sono descritte di seguito:
  \begin{center}
    \includegraphics[scale=0.5]{img/IModel}
  \end{center}
  %--------------------------------------------------------------------------------------------------------------------------------------
  \subsubsection{View}
  Il livello di View è ciò che l'utente vede e con il quale può interagire per inviare comandi al resto del sistema.\\
  L'entità \textbf{IView} rappresenta questo livello, e fornisce le funzionalità sotto descritte per interagire con esso:
  \begin{center}
    \includegraphics[scale=0.5]{img/IView}
  \end{center}
  %--------------------------------------------------------------------------------------------------------------------------------------
  \subsubsection{Controller}
  Il livello Controller è il vero e proprio core del sistema. Esso rappresenta la business logic del sistema, ciò che collega tra loro i due livelli
  sopra citati: Model e View.
  %--------------------------------------------------------------------------------------------------------------------------------------
  \subsection{Design Dettagliato}
  %--------------------------------------------------------------------------------------------------------------------------------------
  \section{Sviluppo}
  %--------------------------------------------------------------------------------------------------------------------------------------
  \section{Note di Sviluppo}
  %--------------------------------------------------------------------------------------------------------------------------------------
  \section{Conclusioni}
  %--------------------------------------------------------------------------------------------------------------------------------------
\end{document}
